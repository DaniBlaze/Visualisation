This report examined two different boarding methods: one where passengers board in blocks and another where they enter the plane in a specific sequence (Steffen J's method). Upon analyzing their strengths and weaknesses, a third model was mostly based on Steffen's method to minimize its biggest inefficiency: the time spent looking for other bins with available space to store the passenger's luggage. This model succeeded to surpass the other two in almost every performance metric used.
Finally, with the help of different scenarios it could be concluded that none of the first two models has a specific feature that dictates a better performance regarding this luggage issue. Also, when the luggage is unevenly distributed along the airplane, this tends to result in a significantly worse performance that, surprisingly, does not worsen other problematic situations such as passengers having to leave their seats in order to let others pass.
Given the strong correlation between the total boarding time and issues related to the storage of luggage and seat interference, it is thus suggested that the performances in these two aspects are key measurements for boarding models.

